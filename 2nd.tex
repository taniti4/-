\section{関連研究}

初等教育において算数文章題を対象とした作間学習は有効な学習形態だと考えられている.しかし,通常ある解法から学習者が作成しうる問題は多様であるため,それぞれが作成した問題に対して個別の対応が必要となる.よって教授者の負荷は大きく,作問学習は有効な学習手段ではあるものの実際の教育現場ではあまり実施されてこなかった\cite{nakano00}.このような問題を解決するため,初等教育における作間学習を実現した研究として,単文統合型作間学習支援システム「モンサクン」に関する研究がある\cite{yoko06}-\cite{yama13b}.モンサクンは算数文章題をモデル化することでその構成要素を単文単位で部品化し,その部品の組み立てを単文統合として行わせることで作問活動を実現し,学習者による算数の文章題に対するモデルの概念獲得や理解の促進を試みたシステムである.この一連の活動を通して,問題の構造を把握する力の獲得を期待している.モンサクンはすでに小学校の授業実践で利用されており,一般的な算数文章題の成績向上の他,情報過剰間題,作間課題,プライミングテストなどの道具的理解に留まらず,関係的理解を必要とするテストの成績向上まで,その効果が確認されている.

モンサクンの作問活動は,算数文章題を三つの単文で構成されるものとした定義に基づいて設計されたものであり,また,その問題の組み立ては,この定義された問題の構造を操作することで行われる.モンサクンでは,算数の文章題の構造的な記述や構造に基づいた問題の部品化を事前に行い,その部品群を課題の要件と共に学習者に提示する.和もしくは差の演算1回で解ける問題を対象とした事例では,一つの問題は三つの問題状況を表す文と一つの定型の問いとして表現されており,問題状況を表す三つの文を組み立てることが学習者の活動となる.学習者は部品のグラフィカルな組み立て,すなわち,単文カードを取捨選択および並べ替えることで作問学習に着手できる.この時,組み立てられた算数文章題の情報構造の自動診断及びその診断結果に基づくフィードバックが学習者の要求に応じて(回答送信時に)可能である.モンサクンで出題する作問課題では,(1)作成する問題の式・物語を指定している問題制約と(2)作問に使用可能なカードを指定している単文カードセット制約の2点が学習者に与えられており,この制約条件を満たすように問題を作成することが暗黙的に要求される制約充足問題となっている.学習者はこの作問課題において問題を作成していき,問題を作成し回答送信したタイミングでシステムによる診断が行われる.正解の場合はそのことを通知し,逆に不正解の場合は誤りに応じたフィードバックをシステム上で学習者に返答する.
