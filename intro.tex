学習を促進する上で試行錯誤的な情報構造操作を行うことは有効であると示唆されている\cite{新井邦二郎1973知的行為の多段階形成理論}\cite{Perpart80}.ここでは人の思考は情報構造に対する操作として捉えることができると仮定されており,このような仮定に基づく学習支援システムの設計開発は「オープン情報構造アプローチ」と呼ばれている\cite{hira0}.オープン情報構造指向アプローチに基づき,平嶋は,学習支援システム研究を(1)利用すべき, あるいは開発すべき情報技術,(2)それによって設計できる, あるいは設計すべきシステム,(3)目指すべき,あるいは目指すことのできる教育・学習活動,の3つの融合として定義し,それぞれを中心とする(1)情報技術シーズベース,(2)情報構造指向,(3)教育ニーズベースの3つに学習支援システムの設計・開発アプローチを分類した\cite{hira1}\cite{hira2}.ところで昨今,教育ビッグデータやLearning Analyticsと呼ばれる枠組みのもとで,学習履歴データの集計と統計による提示により数多くの教育改善が試みられている\cite{uwano}.例えば,脱落のおそれがある学習者を発見し,特定の講座において失敗しないように配慮を受けられるようにしたシステムがある\cite{la1}\cite{la2}.このような統計的手法に基づく学習支援は(1)情報技術シーズベースと考えられている\cite{hayashi1}.(1)情報技術シーズベースの技術と(2)情報構造指向とを融合させた取り組み,すなわち,意味的構造記述に基づく学習活動に由来と性質を持ったデータを確率・統計的機械処理で分析した事例はいくつかの研究成果\cite{hayashi1}で見られるものの,これまで十分に行われていない.

そこで本稿では,「組み立てることによる学習」や「探索的再構成を通した学習」と呼ばれる学習形態\cite{hira2}\cite{miwa}に焦点を当て,確率モデルにより学習ログとして得られる思考経験データから潜在的構造を導き出し,それを活用する方法を提案する.活用法としては,潜在的構造を用いた思考パターンの推定・予測や構成原理に立ち返った解釈などが含まれる.本稿では具体例として,単文統合型作問学習環境モンサクンでの再構成課題に焦点を当て,記録された操作ログから,学習者の特徴を明らかにする.本研究の貢献は,学習者同士の類似度や,各学習者の学習進度や理解度といった情報をデータに基づき定量化することであり,教授者が学習者に対する介入など学習支援を行う際に参考となる情報を提供できるようにする点にある.なお,人の思考の再現には様々なレベルがあると考えられるが,本研究では,人の思考の全てを扱うのではなく,学習課題の情報構造から規定される人の思考のみを対象とする.
