\section{緒言}

地域商業がまちづくりに果たす役割は多機能に渡るとされている.その中で,商店街の維持や振興は,地域経済の発展を考える上で不可欠なことと認識されている.商店街は,経済的側面だけでなく,地域伝統や文化の継承・発展,都市デザインや景観の維持・改善にも不可欠な存在だと言われている.特に,商店街は地域住民が集まる場所であると共に,催事・イベント・祭りの開催といった役割も担ってきた.よって,地域活性化の担い手や地域コミュニティを形成する場としても,商店街は不可欠な存在と言える.その一方,商店街や中心市街地の衰退は,人口減少や少子高齢化の次に続く地域が抱える課題として全国各地で問題視されている.中小企業庁が行った平成30年度商店街実態調査によると,全国の商店街では空き店舗の割合は約14\%であり,53.7\%の商店街は今後も空き店舗が増加するだろうと回答している.また,近年のEC市場の急速な拡大や生活様式の変化などの影響を受け,約70\%の店舗が景況の衰退を感じている.このように,商店街の業況はますます厳しさを増すことが予想されるため,その効果的な対策が望まれている.地域の商工業の活性化を実現するためには,これからの地域社会を担う若年者層の地域に対する興味・関心を促進することが重要となる.そこで本研究では,地域振興の学習をテーマとした地域課題解決型教育を調査することを目的とする.