\section{結言}

本研究では,地域振興策の学習をテーマとした地域課題解決型教育を調査するため,地域の商店街の現状を明らかにし,地域性を的確に把握することの重要性を確認した.%また,若年者を地域活性の鍵とする重要性を示した.
なお,商店街を維持,存続させるためには,センチメンタル価値を高めることが重要とされている.センチメンタル価値とは,街の歴史や愛着心が形成する心理的価値のことであり,存続する商店街は高いセンチメンタル価値を有するとされている\cite{足立基浩2015}.足立は,行動経済的な視点から商店街の衰退及び顧客の減少を理論的に整理し,商店街の維持,存続にはセンチメンタル価値が重要な要素であることを示した.足立はまた,「地域愛」にはまちづくりへの「行動」を刺激する効果があるとした.%り,センチメンタル価値の醸成に正に寄与すると述べている.したがって,今後,まず若者が地域の商店街に目を向ける仕掛けを導入し,
したがって,今後は「地域への愛着」に着眼し,顧客と商店との関係構築に焦点を当てた地域課題解決教育の実践例を調査する予定である.
%支援することは,商店街振興に有効だと考えられる.