\section{関連研究}

地域の商店街を取り巻く環境が厳しさを増す中,商店街振興組合や非営利活動法人が商店街の特色を生かした独自の取り組みを推進し,商店街の課題解決や活性化を実現させた事例はいくつか報告されている\cite{福川2018}.例えば,アートをコンテンツとした愛知県豊橋中心市街地,地域外の店舗を呼び込み商店街を活性化させた新潟県の三条中央商店街,空きビルを再生し利便性を高め来街者を増加させた福岡県北九州市の魚町商店街,デザイン・ビジネス・スキームの3点を組み合わせ集客を向上させた香川県高松丸亀町商店街,高齢社会に対応した買い物代行や出張商店街を展開し成功を収めた埼玉県秩父市のみやのかわ商店街など,様々な取り組みが実践されている.これらの事例では,商店街の特性や地域性に合わせた活性化策が立案・遂行されている点で共通が見られる.したがって,商店街の活性化を図るうえで,まず地域性を的確に把握することが重要だと言える.