\section{分析手順}

先行研究において,試行錯誤的な情報構造の操作を学習者自身が行うことが重要であるとされているが,このような活動を意図して実施された実験的・実践的利用を通じて得られたデータを元に,単文統合型作問演習における単文選択プロセスの分析から作問プロセスのモデルを精緻化し個人適応的な支援として分析結果を活用する.これまでの筆者らによるモンサクンの実践的利用から,経験的観測に過ぎないが学習者は決して闇雲に単文を選択して配置しているのでは無く,何らかの意図に従って操作しているように,また,理解の違いに応じて異なった思考をしているように推察される.そこで,このような経験的観測がデータから裏付け,それを診断で活用するための手続きを設計する.ここで言う診断とは,学習者が回答を送信しそれが誤りのパターンとして観測されたとき,そのパターンに対応する事象にエビデンスを設定して得られる推論結果(正解パターンの事後的な出現確率)を学習者及び教授者に提示することを指す.

まず,データベースの説明をして下さい.

次に分析した手順を書いて下さい.