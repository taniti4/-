\documentclass[chukan]{hitsotsuron} %文書のクラスファイルを指定

%題目の各文字列を指定
\title{単文統合型作問支援環境「モンサクン」における\\学習者の特徴分析に関する研究}

\papertype{令和3年度卒業研究概要}
\author{谷口順哉}
\snumber{BL18064}
\advisor{松本慎平}
\date{2021年5月17日}

\begin{document} %文書本体の開始

%題目を実際に作成
\twocolumn[%
\maketitle
]

\if0

Latexの利用に関する参考URL)

LaTeX - コマンド一覧
http://www1.kiy.jp/~yoka/LaTeX/latex.html

LaTeXコマンドシート一覧
https://members.tripod.com/e_luw/gakko/latex1/l_list.html

\fi

% 緒言
学習を促進する上で試行錯誤的な情報構造操作を行うことは有効であると示唆されている\cite{新井邦二郎1973知的行為の多段階形成理論}\cite{Perpart80}.ここでは人の思考は情報構造に対する操作として捉えることができると仮定されており,このような仮定に基づく学習支援システムの設計開発は「オープン情報構造アプローチ」と呼ばれている\cite{hira0}.オープン情報構造指向アプローチに基づき,平嶋は,学習支援システム研究を(1)利用すべき, あるいは開発すべき情報技術,(2)それによって設計できる, あるいは設計すべきシステム,(3)目指すべき,あるいは目指すことのできる教育・学習活動,の3つの融合として定義し,それぞれを中心とする(1)情報技術シーズベース,(2)情報構造指向,(3)教育ニーズベースの3つに学習支援システムの設計・開発アプローチを分類した\cite{hira1}\cite{hira2}.ところで昨今,教育ビッグデータやLearning Analyticsと呼ばれる枠組みのもとで,学習履歴データの集計と統計による提示により数多くの教育改善が試みられている\cite{uwano}.例えば,脱落のおそれがある学習者を発見し,特定の講座において失敗しないように配慮を受けられるようにしたシステムがある\cite{la1}\cite{la2}.このような統計的手法に基づく学習支援は(1)情報技術シーズベースと考えられている\cite{hayashi1}.(1)情報技術シーズベースの技術と(2)情報構造指向とを融合させた取り組み,すなわち,意味的構造記述に基づく学習活動に由来と性質を持ったデータを確率・統計的機械処理で分析した事例はいくつかの研究成果\cite{hayashi1}で見られるものの,これまで十分に行われていない.

そこで本稿では,「組み立てることによる学習」や「探索的再構成を通した学習」と呼ばれる学習形態\cite{hira2}\cite{miwa}に焦点を当て,確率モデルにより学習ログとして得られる思考経験データから潜在的構造を導き出し,それを活用する方法を提案する.活用法としては,潜在的構造を用いた思考パターンの推定・予測や構成原理に立ち返った解釈などが含まれる.本稿では具体例として,単文統合型作問学習環境モンサクンでの再構成課題に焦点を当て,記録された操作ログから,学習者の特徴を明らかにする.本研究の貢献は,学習者同士の類似度や,各学習者の学習進度や理解度といった情報をデータに基づき定量化することであり,教授者が学習者に対する介入など学習支援を行う際に参考となる情報を提供できるようにする点にある.なお,人の思考の再現には様々なレベルがあると考えられるが,本研究では,人の思考の全てを扱うのではなく,学習課題の情報構造から規定される人の思考のみを対象とする.


% 関連研究
\section{関連研究}

初等教育において算数文章題を対象とした作間学習は有効な学習形態だと考えられている.しかし,通常ある解法から学習者が作成しうる問題は多様であるため,それぞれが作成した問題に対して個別の対応が必要となる.よって教授者の負荷は大きく,作問学習は有効な学習手段ではあるものの実際の教育現場ではあまり実施されてこなかった\cite{nakano00}.このような問題を解決するため,初等教育における作間学習を実現した研究として,単文統合型作間学習支援システム「モンサクン」に関する研究がある\cite{yoko06}-\cite{yama13b}.モンサクンは算数文章題をモデル化することでその構成要素を単文単位で部品化し,その部品の組み立てを単文統合として行わせることで作問活動を実現し,学習者による算数の文章題に対するモデルの概念獲得や理解の促進を試みたシステムである.この一連の活動を通して,問題の構造を把握する力の獲得を期待している.モンサクンはすでに小学校の授業実践で利用されており,一般的な算数文章題の成績向上の他,情報過剰間題,作間課題,プライミングテストなどの道具的理解に留まらず,関係的理解を必要とするテストの成績向上まで,その効果が確認されている.

モンサクンの作問活動は,算数文章題を三つの単文で構成されるものとした定義に基づいて設計されたものであり,また,その問題の組み立ては,この定義された問題の構造を操作することで行われる.モンサクンでは,算数の文章題の構造的な記述や構造に基づいた問題の部品化を事前に行い,その部品群を課題の要件と共に学習者に提示する.和もしくは差の演算1回で解ける問題を対象とした事例では,一つの問題は三つの問題状況を表す文と一つの定型の問いとして表現されており,問題状況を表す三つの文を組み立てることが学習者の活動となる.学習者は部品のグラフィカルな組み立て,すなわち,単文カードを取捨選択および並べ替えることで作問学習に着手できる.この時,組み立てられた算数文章題の情報構造の自動診断及びその診断結果に基づくフィードバックが学習者の要求に応じて(回答送信時に)可能である.モンサクンで出題する作問課題では,(1)作成する問題の式・物語を指定している問題制約と(2)作問に使用可能なカードを指定している単文カードセット制約の2点が学習者に与えられており,この制約条件を満たすように問題を作成することが暗黙的に要求される制約充足問題となっている.学習者はこの作問課題において問題を作成していき,問題を作成し回答送信したタイミングでシステムによる診断が行われる.正解の場合はそのことを通知し,逆に不正解の場合は誤りに応じたフィードバックをシステム上で学習者に返答する.


%実験方法
\section{分析手順}

先行研究において,試行錯誤的な情報構造の操作を学習者自身が行うことが重要であるとされているが,このような活動を意図して実施された実験的・実践的利用を通じて得られたデータを元に,単文統合型作問演習における単文選択プロセスの分析から作問プロセスのモデルを精緻化し個人適応的な支援として分析結果を活用する.これまでの筆者らによるモンサクンの実践的利用から,経験的観測に過ぎないが学習者は決して闇雲に単文を選択して配置しているのでは無く,何らかの意図に従って操作しているように,また,理解の違いに応じて異なった思考をしているように推察される.そこで,このような経験的観測がデータから裏付け,それを診断で活用するための手続きを設計する.ここで言う診断とは,学習者が回答を送信しそれが誤りのパターンとして観測されたとき,そのパターンに対応する事象にエビデンスを設定して得られる推論結果(正解パターンの事後的な出現確率)を学習者及び教授者に提示することを指す.

まず,データベースの説明をして下さい.

次に分析した手順を書いて下さい.

%結論
\section{結言}

本研究では,地域振興策の学習をテーマとした地域課題解決型教育を調査するため,地域の商店街の現状を明らかにし,地域性を的確に把握することの重要性を確認した.%また,若年者を地域活性の鍵とする重要性を示した.
なお,商店街を維持,存続させるためには,センチメンタル価値を高めることが重要とされている.センチメンタル価値とは,街の歴史や愛着心が形成する心理的価値のことであり,存続する商店街は高いセンチメンタル価値を有するとされている\cite{足立基浩2015}.足立は,行動経済的な視点から商店街の衰退及び顧客の減少を理論的に整理し,商店街の維持,存続にはセンチメンタル価値が重要な要素であることを示した.足立はまた,「地域愛」にはまちづくりへの「行動」を刺激する効果があるとした.%り,センチメンタル価値の醸成に正に寄与すると述べている.したがって,今後,まず若者が地域の商店街に目を向ける仕掛けを導入し,
したがって,今後は「地域への愛着」に着眼し,顧客と商店との関係構築に焦点を当てた地域課題解決教育の実践例を調査する予定である.
%支援することは,商店街振興に有効だと考えられる.

\bibliographystyle{jplain}    % スタイル(bstファイル名)を記述する.jplain,plainはpLaTeXに標準で付いている.
%\nocite{*}                 % 本文中で参照していない文献をリストに載せたい場合に用いる.*を指定すると全て載せる.通常使わないのでコメントアウトしている.
\bibliography{reflist}    % bibファイル名を指定する.拡張子は除く.
\end{document} 
